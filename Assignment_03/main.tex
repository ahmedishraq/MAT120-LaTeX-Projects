\documentclass[12pt]{article}
\usepackage{amsfonts}


\usepackage[utf8]{inputenc}
\usepackage{comment}

%\usepackage{pgfplots}
%\pgfplotsset{width=10cm, compat=1.9}
%\documentclass[border=2mm,tikz]{standalone}
%\usetikzlibrary{datavisualization}


\usepackage{fancyhdr}
\usepackage{comment}
\usepackage[a4paper, top=2.2cm, bottom=2.5cm, left=2.2cm, right=2.2cm]%
{geometry}
\usepackage{times}
\usepackage{amsmath}
\usepackage{changepage}
\usepackage{amssymb}
\usepackage{graphicx}%
\setcounter{MaxMatrixCols}{30}
\newtheorem{theorem}{Theorem}
\newtheorem{acknowledgement}[theorem]{Acknowledgement}
\newtheorem{algorithm}[theorem]{Algorithm}
\newtheorem{axiom}{Axiom}
\newtheorem{case}[theorem]{Case}
\newtheorem{claim}[theorem]{Claim}
\newtheorem{conclusion}[theorem]{Conclusion}
\newtheorem{condition}[theorem]{Condition}
\newtheorem{conjecture}[theorem]{Conjecture}
\newtheorem{corollary}[theorem]{Corollary}
\newtheorem{criterion}[theorem]{Criterion}
\newtheorem{definition}[theorem]{Definition}
\newtheorem{example}[theorem]{Example}
\newtheorem{exercise}[theorem]{Exercise}
\newtheorem{lemma}[theorem]{Lemma}
\newtheorem{notation}[theorem]{Notation}
\newtheorem{problem}[theorem]{Problem}
\newtheorem{proposition}[theorem]{Proposition}
\newtheorem{remark}[theorem]{Remark}
\newtheorem{solution}[theorem]{Solution}
\newtheorem{summary}[theorem]{Summary}
\newenvironment{proof}[1][Proof]{\textbf{#1.} }{\ \rule{0.5em}{0.5em}}

\newcommand{\Q}{\mathbb{Q}}
\newcommand{\R}{\mathbb{R}}
\newcommand{\C}{\mathbb{C}}
\newcommand{\Z}{\mathbb{Z}}

\begin{document}

\title{MAT120: Integral Calculus and
Differential Equations \\
BRAC University \\\\
\textbf{Assignment 3}}

\author{Name - Ishraq Ahmed Esha \\ ID - 19301261 \\ Section - 06 \\ Semester- Fall 2020 \\ \textbf{Set- 20}}
\date{\today}
\maketitle
\pagebreak


%%%%%%%%%%%Starting Point%%%%%%%%%%%%%%%

%%%%%MATH 01%%%%%%%%%%%%
\section{Find the arc length of the curve $x=ln(sec\ y)$\ on the interval $y=0$\ and $y=\frac{\pi}{4}$}


%%\begin{align*}%%
    %%x=ln(sec y)\ & & y=0\ and\ y=\frac{\pi}{4}%%%
%%\end{align*}%%

%%%% Solution of Problem 1%%%%
\textbf{Solution}
\begin{align*}
    x&=ln(sec\ y)\\
    \therefore\frac{dx}{dy}&=\frac{d}{dy}(ln(sec\ y))\\
    using\ chain\ rule,\ Let,\ u=sec\ y\ &\  x=ln\ u\\
    \frac{du}{dy}=sec\ y\ tan\ y\ &\ \frac{dx}{du}=\frac{1}{u}\\
    \therefore \frac{dx}{dy}&=\frac{dx}{du}\frac{du}{dy}\\
    &=\frac{1}{u}\ sec\ y\ tan\ y\\
    &=tan\ y\\
    \therefore\ \left(\frac{dx}{dy}\right)^2&=tan^2\ y\\
    Now,\\ s&=\int_{0}^{\frac{\pi}{4}}\sqrt{1+tan^2\ y}\ dy\\
    &=\int_{0}^{\frac{\pi}{4}}\sqrt{sec^2\ y}\ dy & \left[\because\ 1+tan^2\ x = sex^2\ x\right] \\
    &=\int_{0}^{\frac{\pi}{4}}sec\ y\ dy\\
    Let,\\ u&=sec\ y\ tan\ y\\
    \Rightarrow \frac{du}{dy}&=sec\ y\ tan\ y\ +sec^2\ y\\
    \therefore du&= sec\ y\ tan\ y\ +sec^2\ y\ dy\\
    Now,\\ \int{sec\ y}\ \frac{sec\ y\ +\ tan\ y}{sec\ y\ +\ tan\ y}\ dy\ &=\int{\frac{sec^2\ y\ +\ sec\ y\ tan\ y }{sec\ y\ +\ tan\ y}}\ dy\\
    &=\int\frac{du}{u}\\
    &=ln|u|+c\\
    &=ln|sec\ y\ +\ tan\ y|+c\\
    So,\\ \int_{0}^{\frac{\pi}{4}}sec\ y\ dy\
    &=ln\left[sec\ y\ +\ tan\ y\right]_{0}^{\frac{\pi}{4}}\\
    &=ln|sec\ \frac{\pi}{4}+\ tan\ \frac{\pi}{4}| - ln|sec(0) + tan(0)|\\
    &=ln(\sqrt{2}+1)\\
    & & \textsc{[Answer]}
\end{align*}
%%%\pagebreak%%%

%%%%%MATH 02%%%%%%%%%%%%
\section{Evaluate the following indefinite integral by using a trigonometric substitution or otherwise}


\begin{align*}
    \int\frac{1}{\sqrt{4-9x^2}}dx
\end{align*}


%%%% Solution of Problem 2%%%%
\textbf{Solution}
\begin{align*}
    Let,\\ x&= \frac{3u}{2} & \therefore\ u=\frac{3w}{2}\\
    \therefore dx&=\frac{2}{3}du\\\\
    Now,\\\\ \int\frac{1}{\sqrt{2-9x^2}}dx\ &=\int\frac{1}{\sqrt{4-9\left(\frac{2}{3}u\right)^2}}\ \frac{2}{3}du\\
    &=\frac{2}{3} \int\frac{1}{\sqrt{4-9\ \frac{4}{9}u^2}}du\\
    &=\frac{2}{3} \int\frac{1}{\sqrt{4-4u^2}}du\\
    &=\frac{2}{3} \int\frac{1}{2\sqrt{1-u^2}}du\\
    &=\frac{2}{6} \int\frac{1}{\sqrt{1-u^2}}du\\
    &= \frac{1}{3}\ sin^{-1}(u)+c & \left[\because\ \int\frac{1}{\sqrt{1-x^2}}dx\ = sin^{-1}(x)+c\right]\\
    &=\frac{1}{3}\ sin^{-1}\left(\frac{3x}{2}\right)+c\\
     & & \textsc{[Answer]}
\end{align*}
\pagebreak


%%%%%MATH 03 %%%%%%%%%%%%
\section{Find the arc length of the curve\ $x(t)=t^3-4t$\ ,\ $y(t)=t^2-3t$\ for $-2\leq t \leq 2\pi$}


%%%begin{align*}%%
%%    \int_{0}^{\infty}{x^5}{e^{-\frac{x^2}{5}}}dx%%
%%\end{align*}%%


%%%% Solution of Problem 3 %%%%
\textbf{Solution}
\begin{align*}
    \therefore \frac{dx}{dt}=3t^2-4\quad & \therefore \frac{dy}{dt}=2t-3\\
    Now,\\ s=\int_{-2}^{2\pi}\sqrt{\left(\frac{dx}{dt}\right)^2\ +\ \left(\frac{dy}{dt}\right)^2}\ dt\ &=\int_{-2}^{2\pi}\sqrt{(3t^2-4)^2 + (2t-3)^2}\ dt\\
    &=\int_{-2}^{2\pi}\sqrt{9t^4-20t^2-12t+25}\ dt\\
    & & \textsc{[Answer]}
\end{align*}
\pagebreak


%%%%%MATH 04%%%%%%%%%%%%
\section{Find the area of the surface generated by revolving the curve $x=\sqrt[3]{y}$ over the interval $1\leq y \leq 8$\ about the x-axis}


%%5\begin{align*}%%
%%    \int_{0}^{\frac{3\pi}{2}}{sin^6\left(\frac{x}{3}\right)}{cos^4\left(\frac{x}{3}\right)}%%
%%\end{align*}%%


%%%%Solution of Problem 04%%%%
\textbf{Solution}
\begin{align*}
    x&=\sqrt[3]{y}\\
    \therefore y&=x^3\\
    \Rightarrow\frac{dy}{dx}&=\frac{d}{dx}(x^3)\ =3x^2\\
    \therefore\left(\frac{dy}{dx}\right)^2&=9x^4\\\\
    Now,\\\\ s=\int_{1}^{8}2\pi y\sqrt{1+9x^4}\ dx
    &=\int_{1}^{8}2\pi(x^3)\sqrt{1+9x^4}\ dx & & Let,\\
    &=\int_{10}^{36865}\sqrt{1+9x^4}\ \frac{1}{36}\ du & & u=1+9x^4\quad \therefore \frac{du}{dx}=36x^3\\
    &=2\pi\ \frac{1}{36}\int_{10}^{36865}u^{\frac{1}{2}}\ du & & du=36x^3\ dx \quad\therefore \frac{1}{36}=x^3\ dx\\
    &=\frac{2\pi}{36}\ \frac{2}{3}\left[u^{\frac{3}{2}}]\right]_{10}^{36865} & & x\rightarrow1\ u\rightarrow10\ and\ x\rightarrow8\ u\rightarrow36865\\
    &\frac{\pi}{27}\left(36865^{\frac{3}{2}}-10^{\frac{3}{2}}\right)\\
    & & \textsc{[Answer]}
\end{align*}
\pagebreak


%%%%%MATH 05%%%%%%%%%%%%
\section{Find the area of the surface generated by revolving the curve $x=2\sqrt{1-y}$\ over the interval\ $-1\leq y \leq 0$\ about the y axis}


%%\begin{align*}%%
    %%\int_{0}^{3^{\frac{2}{3}}}{x^{\frac{7}{2}}}{\left(3-x^{\frac{3}{2}}\right)^4}dx\\%%
%%\end{align*}%%


%%%%Solution of Problem 05%%%%
\textbf{Solution}
\begin{align*}
    x&=2\sqrt{1-y}\\
    \frac{dx}{dy}&=2\ \frac{(-1)}{2\sqrt{1-y}}\quad =\frac{-1}{\sqrt{1-y}}\\
    \therefore\left(\frac{dx}{dy}\right)^2&=\frac{1}{1-y}\\\\
    Now,\\\\ s=\int_{-1}^{0}2\pi x\sqrt{1+\frac{1}{1-y}}\ dy\ 
    &=\int_{-1}^{0}2\pi x\sqrt{\frac{1-y+1}{1-y}}\ dy &  Let,\\
    &=\int_{-1}^{0}2\pi x\sqrt{\frac{2-y}{1-y}}\ dy & u=2-y\\
    &=2\pi\int_{-1}^{0}2\sqrt{1-y}\frac{\sqrt{2-y}}{\sqrt{1-y}}\ dy & du=-dy\\
    &=-4\pi\int_{3}^{2}u^{\frac{1}{2}}\ du & y\rightarrow-1\ u\rightarrow3\ and\ y\rightarrow0\ u\rightarrow2\\
    &=-4\pi\ \frac{2}{3}\left[u^{\frac{3}{2}}\right]_{3}^{2}\\
    &=-\frac{8\pi}{3}\left(2^{\frac{3}{2}}-3^{\frac{3}{2}}\right)\\
    &=\frac{8\pi}{3}\times(2.36)\\
    & & \textsc{[Answer]}
\end{align*}
\end{document}
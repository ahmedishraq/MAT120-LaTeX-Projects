\documentclass[12pt]{article}
\usepackage{amsfonts}


\usepackage[utf8]{inputenc}
\usepackage{comment}

%\usepackage{pgfplots}
%\pgfplotsset{width=10cm, compat=1.9}
%\documentclass[border=2mm,tikz]{standalone}
%\usetikzlibrary{datavisualization}


\usepackage{fancyhdr}
\usepackage{comment}
\usepackage[a4paper, top=2.2cm, bottom=2.5cm, left=2.2cm, right=2.2cm]%
{geometry}
\usepackage{times}
\usepackage{amsmath}
\usepackage{changepage}
\usepackage{amssymb}
\usepackage{graphicx}%
\setcounter{MaxMatrixCols}{30}
\newtheorem{theorem}{Theorem}
\newtheorem{acknowledgement}[theorem]{Acknowledgement}
\newtheorem{algorithm}[theorem]{Algorithm}
\newtheorem{axiom}{Axiom}
\newtheorem{case}[theorem]{Case}
\newtheorem{claim}[theorem]{Claim}
\newtheorem{conclusion}[theorem]{Conclusion}
\newtheorem{condition}[theorem]{Condition}
\newtheorem{conjecture}[theorem]{Conjecture}
\newtheorem{corollary}[theorem]{Corollary}
\newtheorem{criterion}[theorem]{Criterion}
\newtheorem{definition}[theorem]{Definition}
\newtheorem{example}[theorem]{Example}
\newtheorem{exercise}[theorem]{Exercise}
\newtheorem{lemma}[theorem]{Lemma}
\newtheorem{notation}[theorem]{Notation}
\newtheorem{problem}[theorem]{Problem}
\newtheorem{proposition}[theorem]{Proposition}
\newtheorem{remark}[theorem]{Remark}
\newtheorem{solution}[theorem]{Solution}
\newtheorem{summary}[theorem]{Summary}
\newenvironment{proof}[1][Proof]{\textbf{#1.} }{\ \rule{0.5em}{0.5em}}

\newcommand{\Q}{\mathbb{Q}}
\newcommand{\R}{\mathbb{R}}
\newcommand{\C}{\mathbb{C}}
\newcommand{\Z}{\mathbb{Z}}

\begin{document}

\title{MAT120: Integral Calculus and
Differential Equations \\
BRAC University \\\\
\textbf{Assignment 1}}

\author{Name - Ishraq Ahmed Esha \\ ID - 19301261 \\ Section - 06 \\ Semester- Fall 2020 \\ \textbf{Set- 12}}
\date{\today}
\maketitle

%%%%%%%%%%%Starting Point%%%%%%%%%%%%%%%

%%%%%MATH 01%%%%%%%%%%%%
\section{Evaluate the following integral by interpreting it as area, or otherwise:}


\begin{align*}
    \int _{-1}^5\left|x-3\right|\:
\end{align*}

%%%% Solution of Problem 1%%%%
\textbf{Solution}
\[ x-3 =
  \begin{cases}
    x-3;\ x-3\geq{0};\ x\geq{3}\\
    -(x+3);\ x-3\leq{0};\ x\leq{3}
  \end{cases}
\]\\
\begin{align*}
    \int _{-1}^5\left|x-3\right|\
    &=\int _{-1}^3 (-x+3)\ dx\ +\int_{3}^5 (x-3)\ dx\\
    &=\int _{-1}^3 -x\ dx\ +\ \int _{-1}^3 3\ dx\ +\ \int _{3}^5 x\ dx\ -\ \int _{3}^5 3\ dx\\
    &=\left[\frac{-x^2}{2} \right]_{-1}^3 \ +\ 3x\ +\ \left[\frac{x^2}{2} \right]^5_3\ -\ 3x\\
    &=\left(\frac{-9}{2}+9+\frac{7}{2}\right)\ +\ \left(\frac{25}{2}-15+\frac{9}{2}\right)\\
    &=8+2\\
    &=10
    & & \textsc{[Answer]}
\end{align*}
\pagebreak

%%%%%MATH 02%%%%%%%%%%%%
\section{Evaluate the following indefinite integral with a substitution, or otherwise:}


\begin{align*}
    \int\frac{x}{\sqrt{4-x^2}} & &  \tesxtsc{where (4\geq{x^2})}
\end{align*}


%%%% Solution of Problem 2%%%%
\textbf{Solution}
\begin{align*}
    Let,\\ u&=4-x^2\\
    \frac{du}{dx}&=-2x\\
    du&=-2x\ dx\\
    -2x\dx&=du\\
    x\dx&=\frac{-1}{2}\ du\\
    Now, \\ \int\frac{x}{\sqrt{4-x^2}}\ dx\
    &=\int-\frac{1}{2\sqrt{u}}\ du\\
    &=-\frac{1}{2}\int\frac{1}{\sqrt{u}}+c\\
    &=-\sqrt{u}+c\\
    &=-\sqrt{4-x^2}+c
     & & \textsc{[Answer]}
\end{align*}
\pagebreak


%%%%%MATH 03 (a)%%%%%%%%%%%%
\section{(a) If $I_n = \int{x^{n-1}} {e^x}\ dx$,\ by\ using\ integration\ by\ parts,\ prove\ that}


\begin{align*}
    I_n = \int{x^{n-1}}{e^x}-(n-1)I_{n-1}
\end{align*}


%%%% Solution of Problem 3(a)%%%%
\textbf{Solution}
\begin{align*}
    Here,\\  f(x)&=x^{n-1}         & g(x)&=e^{x}\\
    f{'}(x)&=\frac{d}{dx}(x^{n-1})     & G(x)&=\int{e^x}\ dx\\
    &=(n-1)x^{n-2}               &        &=e^x\\\\
    Now,\\\int {x^{n-1}}{e^x}\ dx\
    &=x^{n-1}e^x-\int{(n-1)x^{n-2}}e^x\ dx\\
    &=x^{n-1}e^x-\ (n-1)\int {x^{n-2}}{e^x}\ dx\\
    &=x^{n-1}e^x-\ (n-1)I_{n-1}\\\\
    \therefore  I_n = \int{x^{n-1}}{e^x}-(n-1)I_{n-1} 
    & & \textsc{[Proved]}
\end{align*}
\pagebreak


%%%%%MATH 03 (b)%%%%%%%%%%%%
\textbf{3 (b) Evaluate $I_1 = \int{e^x}\ dx$ and use the reduction formula in part (a) to calculate $I_3$ }
%%\section{(a) If $I_n = \int{e^x}\ dx$ and use the reduction formula in part (a) to calculate $I_3$ }


%%%% Solution of Problem 3(b)%%%%
\textbf{Solution}
\begin{align*}
    Given,\\ I_n&=\int {x^{n-1}}{e^x}\ dx\\
    I_1&=\int {x^{1-1}}{e^x}\ dx\\
    I_1&=\int {x^0}{e^x}\ dx\\
    I_1&=\int{e^x}\ dx\\
    \therefore I_1=\int e^x\ dx
     & & \textsc{[Answer]}\\\\
     Given\\ I_n&=\int {x^{n-1}}{e^x}\ dx\\
     I_3&=\int {x^{3-1}}{e^x}\ dx\\
    I_3&=\int {x^2}{e^x}\ dx\\
    \therefore I_3=\int{x^2} e^x\ dx\\\
    Now\ using\ Reduction\ Formula,\\
    f(x)&=x^2      &   g(x)=e^{x}\\
    f{'}(x)&=\frac{d}{dx}(x^2)  & G(x)=\int{e^x}\ dx\\
    &=2x                    &   =e^x\\
    So,\\ 
    I_3=\int {x^2}{e^x}\ dx\
    &=x^2e^x-\int{2x}{e^x}\ dx\\
    &=x^2e^x-2xe^x-2e^x   & \left[\because \int {u}{v}\ dx=u\int {v}dx-\int {\frac{du}{dx}}\int {v}dx\ dx\right]\\\\
    \therefore I_3=x^2e^x-2xe^x-2e^x
    & & \textsc{[Answer]}
\end{align*}
\pagebreak

%%%%%MATH 04%%%%%%%%%%%%
\section{Evaluate the following integral by decomposing it into partial fractions, or otherwise:}


\begin{align*}
    \int{\frac{2x+7}{x^2-16x+55}}
\end{align*}


%%%%Solution of Problem 04%%%%
\textbf{Solution}
\begin{align*}
    \int {\frac{2x+7}{x^2-16x+55}} & & &= x^2-16x+55\\
    =\int {\frac{2x+7}{(x-11)(x-5)}} & & &=x^2-11x-5x+55\\
                             & &  &=x(x-11)-5(x-11)\\
                             & &  &=(x-11)(x-5)\\
    Let,\\
    \int {\frac{2x+7}{(x-11)(x-5)}}&=\frac{A}{(x-11)}+\frac{B}{(x-5)}\\\\
    2x+7 &=A(x-5)+B(x-11)\\\\
    Now,\\
    x&=5;\ \ \ 17=-6B\ \ \ \therefore B=-\frac{17}{6}\\
    x&=1;\ \ \ 29=6A\ \ \ \therefore A=\frac{29}{6}\\
    So,\\
     \int {\frac{2x+7}{x^2-16x+55}} &=\frac{29}{6}\int {\frac{1}{(x-11)}} + -\frac{17}{6} \int{\frac{1}{(x-5)}}\\
    &=\frac{29}{6}ln(x-11)-\frac{17}{6}ln(x-5)+c
    & & \textsc{[Answer]}
\end{align*}
\pagebreak


%%%%%MATH 05%%%%%%%%%%%%
\section{Evaluate the following improper integral with the help of Gamma functions, or otherwise:}


\begin{align*}
    \int_{0}^{\infty}{(3t)^7}{e^{(-3t)}}\ dt
\end{align*}


%%%%Solution of Problem 05%%%%
\textbf{Solution}
\begin{align*}
    Let,\\ 3t&=u\\
    \Rightarrow t&=\frac{u}{3}\\
    \Rightarrow \frac{d}{dt}(3t)&=\frac{d}{dt}(u)\\
    \Rightarrow 3&=\frac{du}{dt}\\
    \therefore dt&=\frac{du}{3}\\\\
    Now,\\ \int_{0}^{\infty}{(3t)^7}{e^{(-3t)}}\ dt\
    &=\int_{0}^{\infty}{u^7}{e^{-u}}\ dt\\
    &=\int_{0}^{\infty}{u^7}{e^{-u}}\frac{du}{3}\\
    &=\frac{1}{3}\int_{0}^{\infty}{u^{8-1}}{e^{-u}}du\\
    &=\frac{1}{3}\ \Gamma(8) & \left[\because \Gamma(n)=\int_{0}^{\infty}{x^{n-1}}e^{-x}\ dx\right]\\
    &=\frac{1}{3}\ (7!)\\
    &=1680
    & & \textsc{[Answer]}
\end{align*}
\end{document}